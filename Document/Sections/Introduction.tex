A Brain-Computer Interface (BCI) is a computerised system that allows for communication with the human brain, by interpreting brain activity in real time and a computer system can respond accordingly to a numerous predicated events through the use of commands (\cite{DatasetPaper}). This is highly beneficial for many sectors including healthcare, as for high-functioning patients suffering from Autism Spectrum Disorder (ASD), this can be able to utilise sensory equipment to combat an event by recognising the trigger through deriving the P300 paradigm. The P300 is derived when the subject is provided a stimulus via an event related potential (ERP) component, that occurs as a neural response measured using electroencephalography (EEG), 300ms after the stimulus triggers such an event in the brain, a brain-computer interface can improve the deficit of the central nervous system (CNS) to which in this research is is target towards high-functioning patients suffering from autism spectrum disorder (ASD). The overall aim is to create a real time computer interface that recognises and responds to such an event, the use of classifiers can identify and learn these neural responses. \\

The problem is which classifier is the most appropriate for this dataset as to source the data and train and test the data sourced by electroencephalography (EEG) and accurately determine that a P300 trigger has occurred. Then the computer will respond using a set of pre-defined commands to assist the high-functioning patient who has ASD (\cite{ClinicalTrialPaper}).

\section{Project Objectives}
\label{Project Objectives Section}

The objective of this research is to improve the test accuracy of the work completed by Dr. Ramaswamy (\cite{PalaniPaper}), where the research was completed in 2 phases and gained a best overall accuracy result of 76\%. This research aims to replicate the work already done and experiment with varies variations of the 3 classifiers used: Bayes Linear Discriminant Analysis (BLDA), Convolutional Neural Network (CNN) and Random Undersampling Boosting (RUS). This research project utilises MATLAB instead of Python which is most renowned for its deep learning and machine learning capabilities. The use of MATLAB 2022a is to safeguard the integrity of the research but using the same integrated development environment (IDE) as \cite{PalaniPaper}. \\

The final project objective differs from the original as trouble with runtime limiting the amount of testing that could be completed, however the CNN was experimented with as it was indicated that 'CNN may produce improved results' (\cite{PalaniPaper}) in the conclusion, so to open up for future improvement. The objective in detail is to test variations of the CNN classifier to find a most optimal values and options, changing the filter size and number of filters in the convolutional layers to improve the overall testing accuracy. Testing of the data will identical replicate \cite{PalaniPaper} by testing all 15 subjects and session: 1 to 7, 1 to 3 and 4 to 7 separately as an accuracy of 67\% was provided for all subjects, session 4 to 7 only. The separate review of the session is a secondary objective to indicate which session holds the best accuracy and control over the subject.\\

A summary of the project objectives:

\begin{itemize}
  \item Form a baseline accuracy by replicating the work done in \cite{PalaniPaper} and improve on them and those of the current work.
  \item Test consecutively and record the accuracy of different CNN variations.
  \item Observe and indicate strong sessions and subjects that contribute the most towards the overall accuracy.
  \item Improve the efficiency of the CNN.
\end{itemize}

\section{Project Overview}
\label{Project Overview Section}

The project is set into stages of research (literature and technology), method of approach, testing and results, conclusion of the project and a project discussing with mention to improvements and future work. The literature review in \cref{Literature Review Chapter} contains a brief literature research in which explores the use of BCI's, their applications and the clinical trial dataset used in this project, an extensive review of the technology, in what a CNN is and what are the different layers and options available and a review of the related work completed in (\cite{PalaniPaper}). The methodology in \cref{Methodology Chapter} outlines the preparation taken on the existing files and the modification to the dataset to best suit the use of MATLAB, a detailed review of the CNN MATLAB files used in this project which include a function containing the Convolutional Neural Network (CNN) and 3 testing files which are divided to review the data of all 15 subjects and sessions: 1 to 7, 1 to 3 and 4 to 7. \\

The testing and results in \cref{Testing & Results Chapter} details the variations of the tested CNN and their results which are graphical shown to provide insight on the behaviour of changing the supplement layers, layer values and CNN options. Separating the results into 3 main sections related to the testing of all 15 subjects and sessions: 1 to 7, 1 to 3 and 4 to 7. This will be concluded in \cref{Conclusion Chapter} to provide a summary of the results and completion of the project objectives and how the project contributed to the growing research towards BCI's in healthcare. Adjacent to the conclusion will be a discussion in \cref{Discussion Chapter} to review issues with the project, more so with run time and a discussion of future improvements of the project.