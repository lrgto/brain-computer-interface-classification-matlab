In this project, 59 tests were run, 33 of which were full tests, the others were brief experiments run to observe the run time and test the testing and subject and session files. In the 33 tests, values of a variety of CNN features where changes, the CNN was restructured to provide and increase in the accuracy and proves that the CNN can still be tailored to the dataset. This research project was successful for improving the test accuracy for all the sessions, however as stated, the run time limited the amount of variations of the CNN that could be run, i originally wanted to run each variation 3 times to provide proof that the CNN will replicate itself and showing a pattern amongst the CNN features. MATLAB also required me to physical be at the computer while the test was running as the program would freeze, i only guess that the CPU would fall asleep though i turned off sleep and screensaver modes. Even though MATLAB supports Mac, it's not native, the use of a windows PC was also sourced but the run time remained unchanged. \\

The run time also hindered my testing of a graphing function that when the CNN function is complete the necessary graphs are also made alongside the CNN results, however the filename would need changing manually to avoid same name conflicts. This could not be done as i was not able to use MATLAB while the CNN was running, though could have been done using python instead but calling the function via MATLAB would be more efficient. Another function i was hoping to implement was that of writing the files on one spreadsheet, with the accuracy, run time and predicated object data and the testing file appends to the spreadsheet instead of writing over it. I found while researching this that it's possible but as the testing file closes it wipes the data given to append so in the new CNN test, the append didn't know where to start and finish, Future work on this could be that instead of append the testing file itself, append the SBJ-S.txt files by added the necessary values to the cell and matrix range so that testing file write to empty cells. \\

Future improvement would consist of analysis of Bayes Linear Discriminant Analysis (BLDA) and Random Under Sampling boosting (RUS) with respect to this dataset and Dr Ramaswamy's work in \cite{PalaniPaper}. Further experimentation of the Convolutional Neural Network (CNN) has shown to improve the accuracy with this research paper and it could still be improved through testing and analysis of the CNN's features, variables and values. If this research was redone, keep the max epochs at a value of 30 would suffice, to provide a median of the run time which is half that of a CNN with 50 epochs. By forming a baseline such as was done here and test around it accordingly. If possible run this project on multiple devices, while the CNN is running, the BLDA or RUS classifiers can be tested too.