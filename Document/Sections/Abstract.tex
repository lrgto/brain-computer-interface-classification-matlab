This paper aims to improve a current paradigm, the accuracy within a P300 based brain computer interfaces (BCI), which uses a P300 dataset of 15 high-functioning patients who have Autism Spectrum Disorder (ASD), each patient underwent 7 sessions which extracted electroencephalogram (EEG), P300 based data. This research seeks to improve a current convolutional neural network (CNN) through pre-processing, training and testing the data by running multiple variations of the CNN to best fit the dataset. Two accuracy’s are given to improve (\cite{PalaniPaper}): 67\%, an average of all subjects of sessions 4 to 7 and 76\%, an average of the highest accuracy session of each subject. It was found that one accuracy was improved, the accuracy from 76\% to 84\% by restructuring the dropout layers, decreasing its probabilities, the initial learning rate and the output size of the fully connected layer. The highest average accuracy achieved for session 4 to 7 was 64\%, 3\% under the objective accuracy due to computer and runtime complications of the amount of variations that could be explored, hindering a repetitive runs to forms pattern. It was also concluded that this CNN is subject dependent, in which subjects 2, 4, 8, 10 and 15 showed the highest accuracy’s of their sessions, indicating these individuals hold a greater influence on the dataset. Also proven is that this CNN could have be tailored to the dataset further to increase the accuracy and more so henceforth.
